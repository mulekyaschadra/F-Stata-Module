% Options for packages loaded elsewhere
\PassOptionsToPackage{unicode}{hyperref}
\PassOptionsToPackage{hyphens}{url}
%
\documentclass[
]{book}
\usepackage{amsmath,amssymb}
\usepackage{lmodern}
\usepackage{iftex}
\ifPDFTeX
  \usepackage[T1]{fontenc}
  \usepackage[utf8]{inputenc}
  \usepackage{textcomp} % provide euro and other symbols
\else % if luatex or xetex
  \usepackage{unicode-math}
  \defaultfontfeatures{Scale=MatchLowercase}
  \defaultfontfeatures[\rmfamily]{Ligatures=TeX,Scale=1}
\fi
% Use upquote if available, for straight quotes in verbatim environments
\IfFileExists{upquote.sty}{\usepackage{upquote}}{}
\IfFileExists{microtype.sty}{% use microtype if available
  \usepackage[]{microtype}
  \UseMicrotypeSet[protrusion]{basicmath} % disable protrusion for tt fonts
}{}
\makeatletter
\@ifundefined{KOMAClassName}{% if non-KOMA class
  \IfFileExists{parskip.sty}{%
    \usepackage{parskip}
  }{% else
    \setlength{\parindent}{0pt}
    \setlength{\parskip}{6pt plus 2pt minus 1pt}}
}{% if KOMA class
  \KOMAoptions{parskip=half}}
\makeatother
\usepackage{xcolor}
\usepackage{color}
\usepackage{fancyvrb}
\newcommand{\VerbBar}{|}
\newcommand{\VERB}{\Verb[commandchars=\\\{\}]}
\DefineVerbatimEnvironment{Highlighting}{Verbatim}{commandchars=\\\{\}}
% Add ',fontsize=\small' for more characters per line
\usepackage{framed}
\definecolor{shadecolor}{RGB}{248,248,248}
\newenvironment{Shaded}{\begin{snugshade}}{\end{snugshade}}
\newcommand{\AlertTok}[1]{\textcolor[rgb]{0.94,0.16,0.16}{#1}}
\newcommand{\AnnotationTok}[1]{\textcolor[rgb]{0.56,0.35,0.01}{\textbf{\textit{#1}}}}
\newcommand{\AttributeTok}[1]{\textcolor[rgb]{0.77,0.63,0.00}{#1}}
\newcommand{\BaseNTok}[1]{\textcolor[rgb]{0.00,0.00,0.81}{#1}}
\newcommand{\BuiltInTok}[1]{#1}
\newcommand{\CharTok}[1]{\textcolor[rgb]{0.31,0.60,0.02}{#1}}
\newcommand{\CommentTok}[1]{\textcolor[rgb]{0.56,0.35,0.01}{\textit{#1}}}
\newcommand{\CommentVarTok}[1]{\textcolor[rgb]{0.56,0.35,0.01}{\textbf{\textit{#1}}}}
\newcommand{\ConstantTok}[1]{\textcolor[rgb]{0.00,0.00,0.00}{#1}}
\newcommand{\ControlFlowTok}[1]{\textcolor[rgb]{0.13,0.29,0.53}{\textbf{#1}}}
\newcommand{\DataTypeTok}[1]{\textcolor[rgb]{0.13,0.29,0.53}{#1}}
\newcommand{\DecValTok}[1]{\textcolor[rgb]{0.00,0.00,0.81}{#1}}
\newcommand{\DocumentationTok}[1]{\textcolor[rgb]{0.56,0.35,0.01}{\textbf{\textit{#1}}}}
\newcommand{\ErrorTok}[1]{\textcolor[rgb]{0.64,0.00,0.00}{\textbf{#1}}}
\newcommand{\ExtensionTok}[1]{#1}
\newcommand{\FloatTok}[1]{\textcolor[rgb]{0.00,0.00,0.81}{#1}}
\newcommand{\FunctionTok}[1]{\textcolor[rgb]{0.00,0.00,0.00}{#1}}
\newcommand{\ImportTok}[1]{#1}
\newcommand{\InformationTok}[1]{\textcolor[rgb]{0.56,0.35,0.01}{\textbf{\textit{#1}}}}
\newcommand{\KeywordTok}[1]{\textcolor[rgb]{0.13,0.29,0.53}{\textbf{#1}}}
\newcommand{\NormalTok}[1]{#1}
\newcommand{\OperatorTok}[1]{\textcolor[rgb]{0.81,0.36,0.00}{\textbf{#1}}}
\newcommand{\OtherTok}[1]{\textcolor[rgb]{0.56,0.35,0.01}{#1}}
\newcommand{\PreprocessorTok}[1]{\textcolor[rgb]{0.56,0.35,0.01}{\textit{#1}}}
\newcommand{\RegionMarkerTok}[1]{#1}
\newcommand{\SpecialCharTok}[1]{\textcolor[rgb]{0.00,0.00,0.00}{#1}}
\newcommand{\SpecialStringTok}[1]{\textcolor[rgb]{0.31,0.60,0.02}{#1}}
\newcommand{\StringTok}[1]{\textcolor[rgb]{0.31,0.60,0.02}{#1}}
\newcommand{\VariableTok}[1]{\textcolor[rgb]{0.00,0.00,0.00}{#1}}
\newcommand{\VerbatimStringTok}[1]{\textcolor[rgb]{0.31,0.60,0.02}{#1}}
\newcommand{\WarningTok}[1]{\textcolor[rgb]{0.56,0.35,0.01}{\textbf{\textit{#1}}}}
\usepackage{longtable,booktabs,array}
\usepackage{calc} % for calculating minipage widths
% Correct order of tables after \paragraph or \subparagraph
\usepackage{etoolbox}
\makeatletter
\patchcmd\longtable{\par}{\if@noskipsec\mbox{}\fi\par}{}{}
\makeatother
% Allow footnotes in longtable head/foot
\IfFileExists{footnotehyper.sty}{\usepackage{footnotehyper}}{\usepackage{footnote}}
\makesavenoteenv{longtable}
\usepackage{graphicx}
\makeatletter
\def\maxwidth{\ifdim\Gin@nat@width>\linewidth\linewidth\else\Gin@nat@width\fi}
\def\maxheight{\ifdim\Gin@nat@height>\textheight\textheight\else\Gin@nat@height\fi}
\makeatother
% Scale images if necessary, so that they will not overflow the page
% margins by default, and it is still possible to overwrite the defaults
% using explicit options in \includegraphics[width, height, ...]{}
\setkeys{Gin}{width=\maxwidth,height=\maxheight,keepaspectratio}
% Set default figure placement to htbp
\makeatletter
\def\fps@figure{htbp}
\makeatother
\setlength{\emergencystretch}{3em} % prevent overfull lines
\providecommand{\tightlist}{%
  \setlength{\itemsep}{0pt}\setlength{\parskip}{0pt}}
\setcounter{secnumdepth}{5}
\usepackage{booktabs}
\ifLuaTeX
  \usepackage{selnolig}  % disable illegal ligatures
\fi
\usepackage[]{natbib}
\bibliographystyle{plainnat}
\IfFileExists{bookmark.sty}{\usepackage{bookmark}}{\usepackage{hyperref}}
\IfFileExists{xurl.sty}{\usepackage{xurl}}{} % add URL line breaks if available
\urlstyle{same} % disable monospaced font for URLs
\hypersetup{
  pdftitle={Stata-Support de formation},
  pdfauthor={Mulekya Schadra},
  hidelinks,
  pdfcreator={LaTeX via pandoc}}

\title{Stata-Support de formation}
\author{Mulekya Schadra}
\date{2022-06-20}

\begin{document}
\maketitle

{
\setcounter{tocdepth}{1}
\tableofcontents
}
\hypertarget{about}{%
\chapter*{About}\label{about}}
\addcontentsline{toc}{chapter}{About}

Ce document est écrit comme support de formation dans le logiciel STATA. une formation rélisé pour le Docteur Franc Lutu.
Nous ne prétendons pas aborder toutes les connaissances disponibles dans STATA,
neanmoins nou proposons les compétences éssentieles dans les aspects d'analyse des données.

ce livre est téléchargeable en format pdf ,sur le compte Github ci-sessous \url{https://github.org/mulekya_schadra/}.

\texttt{Ce\ support\ est\ écrit\ dans\ le\ cadre\ d\textquotesingle{}apprentissage\ du\ logiciel\ STATA,\ \ dans\ ce\ livre,\ les\ chapitres\ sont\ organisées\ de\ manière\ à\ inculter\ une\ certaine\ compétence\ dans\ l\textquotesingle{}analyse\ des\ donnes\ Ce\ cours\ est\ \ reparti\ dans\ 5\ chapitres,\ concernat\ les\ aspects\ de\ base\ (}\#`) per

Nous nous basons sur la comprehension progressive. ce sont les bases qui determinent la comprehension des notions suivantes. `

\hypertarget{pries-en-main-du-logiciel}{%
\chapter{Pries en main du logiciel}\label{pries-en-main-du-logiciel}}

le logiciel est un programme de l'entreprise staa utilisé dans le domaine de l'économie et de l'économétrie dans le cadre d'analyse des données.

Ce logiciel est manipulable sous deux angles :

\begin{itemize}
\tightlist
\item
  Interface graphique ;
\item
  Intgerface de commande
\end{itemize}

cette aspect des chose rend le logiciel Stata fléxible quand aux éxigences du moment: \texttt{la\ reproductibilité\ du\ travail\ dans\ l\textquotesingle{}analyse\ des\ données}

\hypertarget{pruxe9sentation-de-linterface}{%
\section{Présentation de l'interface}\label{pruxe9sentation-de-linterface}}

Voici comment resemble l'interface Stata à l'ouverture du programme:

\begin{figure}
\centering
\includegraphics{"fenetre Stata.jpg"}
\caption{Presentation de l'interface Stata}
\end{figure}

4 fenetres principales dont :

\begin{itemize}
\tightlist
\item
  La visionneuse des resultats
\item
  La partie Commande
\item
  la Vue des variables
\item
  L'historique des commandes exécutées.
\end{itemize}

\begin{enumerate}
\def\labelenumi{(\arabic{enumi})}
\tightlist
\item
  Le visionneuse des resultats sert à visualiser les résultats après exécution d'une quelconque tâche dans le cadre du travail sous Stata
\item
  La partie Commande est la partie où on entre du code dans la syntax appropriée à stata, et selon la tâche que l'on souhaiterais éxécuter
\item
  La partie vue des variables quand à elle, sert à montrer le nom des variables contenues dans la base des données et leurs caractéristiques tel que: le type des variables, leur format, les label. Les autres details sont affciché dans la fenêtre juste en bas: la partie propriété des variables
\item
  La fênetre history quand à elle, sert montre l'ensemble des codes exécutées dans la sessions Stata dépuis le début du travail.
\end{enumerate}

à part ces interfaces, nous avons la base à outils et la bare des menus dans stata.

Avec exécution que ce soit par l'utilisation de l'intrerface graphique ou de la partie commande, tout travail passe par l'invité de commande.

\emph{Stata étant un logiciel dédié à l'analyse des données nous allons passer direcetement par la partie qui consiste à charger une base des données dans le mémoire Stata: \texttt{Importation}.}

\hypertarget{definition-de-lespace-de-travail}{%
\section{Definition de l'espace de travail}\label{definition-de-lespace-de-travail}}

En anglais \emph{working direcrorie} est le dossier de lecture et d'écriture d'un programme par défaut. Pour toute session de stata, le repertoire (directory) est les dossier \emph{mes documents}. celui-ci est changé par la commande \textbf{cd} pour signifier \emph{change directorie} et suivi du chemin d'accès complet à ce repertoire.

pour connaitre le chemin d'accès ç un repertoire donnée, il suffit de se selectionner ce dossier et de faire menu contextuelle tout en appuyant le bouton shift du clavier et choisir l'option \emph{copier en tant que chemin d'accès} et ensuite coller dans stata après le mot clé \emph{cd}.

\begin{Shaded}
\begin{Highlighting}[]
  \CommentTok{\# cd "E:\textbackslash{} MES CONSULTANCES\textbackslash{}Dr Franck\textbackslash{}Stata Learning" pour spécifier le }
  \CommentTok{\#  dossier *Stata learning comme repoirtoire de travail pour la }
  \CommentTok{\# session stata. Changer les *\textbackslash{}* en */*.}
\end{Highlighting}
\end{Shaded}

\texttt{Exemlple:\ cd\ "E:/MES\ CONSULTANCES/Dr\ Franck/Stata\ Learning"\ pour\ spécifier\ le\ dossier\ *Stata\ learning\ comme\ repoirtoire\ de\ travail\ pour\ la\ session\ stata.}

\hypertarget{fenetres-aditionnelles-de-stata}{%
\section{fenetres aditionnelles de Stata}\label{fenetres-aditionnelles-de-stata}}

\begin{enumerate}
\def\labelenumi{(\arabic{enumi})}
\tightlist
\item
  Data editor / Data browser pour visualiser les données chargée dans la mémoire sous forme de tableau de manière à faciliter leurs lectures comme s'il s'agissait d'un tableaur (excel par exemple).
  Data bowser est différent de data editor dans ce ses que ce premier permet de visualiser les données sans possibilité de modification, tandis que la le data editor quand à lui offre des possibilité de modification comme dans un tableur classique.
\item
  Graph editor : pour visualiser les graphiques tracés dans stata, en permettant une certaine modification des élèments graphiques tel que le titre, le titre des axes, la vouleur des textes , \ldots{}
\item
  Variable manager qui permmet de visualiser et même des modofier les propriétes des variables contenue dans la base des données stata
\item
  help : pour voir l'aide sur différentes opérations sous stata
\end{enumerate}

Ainsi donc, les points 1,2 et 4 font parti des fenêtres du type \texttt{viewer} dans \texttt{stata}.

\hypertarget{importation-de-la-base-des-donnuxe9es}{%
\section{importation de la base des données}\label{importation-de-la-base-des-donnuxe9es}}

Stata offre plusieures possibilitées de lire les bases des données provenant de plusieures sources externes sont \emph{excel, spss, sas,csv, \ldots{}}.
l'extention des bases des données propre à stata sont les fichiers \emph{.dta}.

pour importer une base des données sous stata, il faut utiliser la fonctiuon reab avec l'extention du fichier.

\begin{enumerate}
\def\labelenumi{(\arabic{enumi})}
\tightlist
\item
  pour un fichier excel: read\_excel avec le chemin d'accès complet du fichier, écrit sous forme de caractète .
\item
  pour un fichier spss, la comande a comme mot clé \emph{read\_spss}
  ainsi de suite
\end{enumerate}

Dans le cadre de ce cours nous allons plus utiliser les fichiers venant de l'excel. Ainsi donc, nous exploiterons plus la fonction \emph{read\_excel} et les différents arguments qui viennent avec.
Specifier la feuille qui contient nos données
spécifier les noms des variables à la première ligne ou pas
spécifier si toutes les données sont importées comme des chaines des caractères ou pas.

\hypertarget{fichier-do-do-file}{%
\section{Fichier do (do-file)}\label{fichier-do-do-file}}

Le dofile est est dichier dans lequel sont stockés les différentes commandes des stata, que l'on pourra exécuter plus tard , au besoin, pour des raisons de continuité et de reproductibilité du travail d'analyse des données.

Notons que k'on peut choisir d'utiliser stata par son interface graphique que par sa partie commande.

\hypertarget{data-cleaning}{%
\chapter{Data CLeaning}\label{data-cleaning}}

\hypertarget{introduction}{%
\section{Introduction}\label{introduction}}

Pourquoi manipuler les données en Stata et pas en Excel ? La raison est simple : pas mal des commandes que l'on va voir ci-dessous existent aussi en Excel et sont certes quelquefois plus simples (si on arrive à les trouver), mais par contre on perd vite le fil de ce que l'on a fait subir aux données avant de passer à l'estimation, et c'est parfois là que se cachent soit les quelques erreurs à l'origine de résultats grotesques soit, au contraire, les mauvais traitements infligés aux chiffres pour obtenir le résultat désiré.

Avec Stata, on peut garder la trace de toutes les manipulations dans le do-file. Celui-ci doit contenir toutes les commandes permettant de passer du fichier-données brut à celui qui est prêt à l'estimation. Il est alors facile de retrouver l'erreur qui tue ou bien de vérifier ce que les chiffres ont subi entre les mains du bourreau avant d'avouer.

La manipulation des données sous stata consiste à

\begin{itemize}
\tightlist
\item
  Typage des variable
\item
  remplaceme nt des valeurs manquantes
\item
  remplacement de certaines variables sous certaines conditions
\item
  codification des variable
\item
  recodage des variables
\end{itemize}

Ainsi, dans uune base des donénes, avant de commencer le nettoyage de la base des donnée sil surffit d'avoir une vue gloobale sur cette base en conaissant les caractéristiques générales de différnetes variables contenues dans la base des données.
Ainsi, nous utilisons les commandes suivantes :

\begin{enumerate}
\def\labelenumi{(\arabic{enumi})}
\tightlist
\item
  \textbf{describe} : permet de décrire toutes les variables de la base des données chargée en mémoire. il nous amène en sortie: le nombre des observation, le nombre des variables, les noms des variables , les labeks et les types de chaque variable sous forme de tableau. //Avec les options \emph{short} pour affichier le nom des variables, \emph{simple} pour affcicher le nombre des variables et le nombre d'observations dans la BD.
\item
  \textbf{codebook} pour voir les différentes ecaractéristiques des variables dans la base des données. utiliser codebook suivi du nom de la variable pour ne voir que les caractéristiques d'une seule variable ou une liste des variables.
\item
  Visualisation de la BD sous forme de tableau

  \begin{itemize}
  \tightlist
  \item
    \emph{browse} pour affcicher uniquement;
  \item
    \emph{edit} pour pouvoir modifier manuellement les valeurs dans la base.
  \end{itemize}
\end{enumerate}

\hypertarget{les-commandes-de-base}{%
\section{Les commandes de base}\label{les-commandes-de-base}}

\hypertarget{la-synthaxe-des-commandes-stata}{%
\subsection{La synthaxe des commandes stata}\label{la-synthaxe-des-commandes-stata}}

Stata comme tous les logiciels, utilise un langage qui n'est ni de
l'anglais, ni du français, mais son propre langage. Certes, les
mots sont empruntés à la langue de Shakespeare, mais la syntaxe
n'a rien à voir avec l'anglais. Hormis quelques exceptions, la
syntaxe des commandes de Stata est:

\texttt{{[}by\ listevar:{]}\ commande\ {[}listevar{]}\ {[}=exp{]}\ {[}if\ exp{]}\ {[}in\ intervalle{]}\ {[}pondération{]}\ {[},\ options{]}}

Le nom de la commande est évidemment obligatoire, et il peut
éventuellement être précédé d'un préfixe by, et le plus souvent il
est suivi d'un ou de plusieurs suffixes. Dans le cas de
commandes particulièrement usuelles, il peut parfois être abrégé,
comme par exemple d pour describe. Les suffixes sont entourés de crochets pour indiquer leur caractère optionnel:
listevar correspond à une liste de variables, exp à une
expression logique, intervalle à une série d'observations
dans le fichier de données, et pondéra t i on à une expression
indiquant la variable et le mode de pondération des données.
Enfin, après une virgule, on peut ajouter une ou plusieurs
options pour l'exécution de la commande.
La syntaxe complète pour chaque commande figure dans les
manuels de référence de Stata, qui restent de ce point de vue
irremplaçables. Mais puisque le préfixe by et les suffixes if, in
et la pondération sont communs à la majorité des commandes,
nous nous en tiendrons dans les chapitres suivants à exposer la
syntaxe de base qui prend la forme: .

\texttt{commande\ {[}listevar{]}\ {[}=exp{]}\ {[},\ options{]}}

Immédiatement après le nom de la commande, une liste de
variables indique sur quelles variables doit s'effectuer la commande. Pour explorer le fichier « census.dta », on tapera:

\texttt{list\ state\ region\ pop}

\begin{enumerate}
\def\labelenumi{(\alph{enumi})}
\item
  Le préfixe \textbf{by}
  permet d'exécuter la commande pour chaque
  sous-ensemble d'observations défini pour chaque valeur de
  listevar. Avant d'exécuter la commande, le fichier doit
  d'abord être trié (avec la commande \textbf{sort} listevar) selon la
  même variable utilisée par le préfixe by. Par exemple, on aura:
\item
  Le suffixe \emph{{[}in intervalle{]}}
  Le suffixe in est moins courant dans la pratique, car il suppose
  de bien connaître l'ordre dans lequel sont classées les
  observations du fichier. TI permet d'exécuter la commande pour
  certaines observations, par exemple:
\end{enumerate}

\begin{Shaded}
\begin{Highlighting}[]
\CommentTok{\# sort region}
\CommentTok{\# by region: list region state pop medage}
\end{Highlighting}
\end{Shaded}

\begin{enumerate}
\def\labelenumi{(\alph{enumi})}
\setcounter{enumi}{1}
\tightlist
\item
  Le suffixe \emph{{[}if exp{]}}
  Le suffixe if restreint l'exécution de la commande au sousensemble
  des observations pour lesquelles l'expression logique
  exp est vraie, c'est-à-dire différente de la valeur O.
  Nous reviendrons dans la section consacrée aux calculs sur la
  manipulation de ces expressions logiques, dites encore
  booléennes. Pour l'heure, un exemple suffit à comprendre le
  fonctionnement de ce suffixe:
\end{enumerate}

\texttt{On\ préférera\ toujours\ sélectionner\ un\ sous-ensemble\ d\textquotesingle{}observation\ avec\ le\ suffixe\ if\ en\ fonction\ de\ variables\ bien\ connues\ et\ qui\ font\ sens,\ plutôt\ que\ de\ se\ fier\ à\ un\ ordre\ arbitraire\ des\ observations\ dans\ le\ fichier.}

\hypertarget{les-commandes-de-depart}{%
\subsection{Les commandes de depart}\label{les-commandes-de-depart}}

\begin{enumerate}
\def\labelenumi{(\arabic{enumi})}
\tightlist
\item
  \textbf{import} : charger la base des donnes dans la mémoire. Suivi de type des fichier. et le chemin d'accès du fichier
\item
  \textbf{clear} vide la mémoire
\item
  \textbf{use} au lieu de mettre tout le sentier. Ne pas oublier de mettre les guillemets comme ils sont (noter le sens !).
\item
  \textbf{save} La commande \texttt{save\ datafile1.dta} est très importante : elle sauvegarde le fichier-données \emph{(.dta)} modifié par le programme sous un autre nom que le fichier initial, ce qui permet de laisser ce dernier intouché. Sinon on altère le fichier initial de façon permanente, ce qui est en général un désastre. - De façon générale, les guillemets (comme dans cd ``c:/path/directory'') sont obligatoires quand les noms spécifiés ne sont pas liés en un seul mot ; par exemple, Stata comprend use ``le nom que je veux.dta'' mais pas use le nom que je veux.dta.
\item
  \textbf{Describe} pour decrire la base des données
\end{enumerate}

\hypertarget{creation-et-correction-des-variables}{%
\subsection{Creation et correction des variables}\label{creation-et-correction-des-variables}}

\begin{enumerate}
\def\labelenumi{(\arabic{enumi})}
\tightlist
\item
  Les commandes \emph{generate} et \emph{replace}
  La commande generate crée de nouvelles variables. Elle a la syntaxe de base suivante:
  \emph{{[}by listevar:{]} generate var = exp{[}if exp{]} {[}in intervalle{]}}
  La commande replace utilise la même syntaxe, sauf qu'elle s'applique aux variables déjà existantes.
\end{enumerate}

Comme on le voit, cette syntaxe est simple, ce qui n'est pas le cas de la forme que peut prendre exp. La première expression exp (après le signe =) spécifie le contenu de la variable, c'est-àdire le plus souvent une valeur numérique. La seconde expression exp (après if) doit être formulée comme une expression logique dont le résultat est soit vrai soit faux: la création (ou le remplacement) de la variable est restreint aux observations pour lesquelles le résultat de l'expression est vrai. Cela n'a l'air de rien, mais la confusion entre les deux expressions est certainement l'erreur la plus fréquente que peuvent faire les utilisateurs de Stata.

\begin{enumerate}
\def\labelenumi{(\arabic{enumi})}
\setcounter{enumi}{1}
\tightlist
\item
  Les opérateurs
\end{enumerate}

Les opérateurs arithmétiques de Stata sont bien classiques:
+ (addition), - (soustraction), * (multiplication), 1 (division),
A (puissance), tout comme les opérateurs relationnels
\textgreater{} (supérieur), \textless{} (inférieur), \textgreater= (supérieur ou égal), \textless= (inférieur
ou égal).

C'est peut-être moins le cas des opérateurs relationnels == (égal)
ou -= (différent, que l'on peut écrire aussi! =), et des opérateurs
logique \&. (et), 1 (ou bien), et - (non).

En effet, Stata distingue le signe = (affectation d'une valeur) du
signe == (égalité entre deux valeurs). Dans le cas d'une
affectation d'une valeur à une variable, la variable apparaît à
gauche du signe = tandis que la valeur affectée apparaît à droite:

\begin{enumerate}
\def\labelenumi{(\arabic{enumi})}
\setcounter{enumi}{2}
\tightlist
\item
  Les expressions logiques dans R
\end{enumerate}

Les expressions logiques sont particulièrement utiles pour créer
des variables dichotomiques, c'est-à-dire qui ne prennent que
deux valeurs, 0 et 1. En effet, une expression logique, c'est-àdire
une expression où interviennent les opérateurs relationnels
\textgreater, \textless, \textgreater=, \textless=, ==, -=, !=, ou bien les opérateurs logiques \&., 1, et
-, est codée 1 lorsque son résultat est vrai, et codée 0 lorsque
son résultat est faux.

La commande tabulate possède une option generate ( ) bien
pratique pour créer une série de variables dichotomiques à partir
d'une variable polytomique. Exécutez la série de commandes:

\textbf{La commande drop}

\hypertarget{les-statisiques-simples-et-leurs-representation-graphiques}{%
\chapter{Les Statisiques Simples et leurs representation Graphiques}\label{les-statisiques-simples-et-leurs-representation-graphiques}}

(Statistiques Univariées)

Avant de mener des analyses à l'aide de modèle de régression et
autres statistiques complexes, il est préférable de tirer le
maximum de l'exploration des données et de statistiques
simples. Cela a deux avantages:

\begin{itemize}
\tightlist
\item
  permettre de mieux connaître les données et donc de repérer
  leurs particularités et leurs éventuelles incohérences, ce qui
  pourra servir pour des analyses statistiques plus
  approfondies;
\item
  permettre de sélectionner des indices et des graphiques
  simples qui rendent le mieux compte des données afin de les
  restituer à un large public: les connaissances en statistique de
  la plupart des mortels ne dépassent guère le pourcentage, et
  de toute façon, même un public de spécialistes ne retiendra en
  définitive que les indices et les graphiques les plus simples.
\end{itemize}

Stata offre de nombreuses commandes pour l'analyse
exploratoire des données, autant sous forme de tableaux que de
graphiques. Comme dans les chapitres précédents, nous
utiliserons le fichier « census.dta » pour illustrer ces
commandes.

La commande \emph{codebook} permet de faire le tri à plat de la base des données en montrant les sttatisques simples et univariées. Et montre toute les informations nécéssaires à la compréhention de la structure d'une variable.

La commande \emph{summarize} listvar permet aussi de résumer
la distribution, en particulier pour les variables numériques
continues. Cela n'aurait pas grand sens, par exemple, de calculer
la moyenne d'une variable discrète.

L'option detail pennet une description plus precIse des
variables continues, incluant les pourcentiles, les quatre plus
grandes (Largest) et plus basses (Smallest) valeurs, ainsi
qu'un indice de dissymétrie (la valeur de Skewness est °pour
la distribution nonnale) et de concentration (la valeur de
Kurtosis est de 3 pour la distribution normale.

À l'inverse de la commande swmnarize, la commande tabulate est utile pour les variables discrètes.

On remarque avec l'option nolabel (pour afficher les codes plutôt que les libellés), que les régions sont classées selon leur numéro de code:

\hypertarget{tableaux-croisuxe9s-uxe0-deux-variables}{%
\section{Tableaux croisés à deux variables}\label{tableaux-croisuxe9s-uxe0-deux-variables}}

La commande tabulate devient vraiment intéressante pour croiser les distributions de deux variables discrètes. La syntaxe de base de cette commande est:

\texttt{tabulate\ varligne\ varcol{[},\ cell\ column\ row\ missing\ nofreq\ wrap\ nolabel\ \textasciitilde{}ll\ chi2\ exact\ gamma\ lrchi2\ iaub\ v{]}}

Les modalités de la première variable citée figurent en ligne, tandis que les modalités de la deuxième apparaissent en colonne. Des options permettent d'obtenir les pourcentages en ligne (row) , en colonne (column) ou par cellule (cell) du tableau:

Pour afficher les pourcentages sans les fréquences, on utilisera l'option nofreq :

\hypertarget{tableaux-croisuxe9s-uxe0-trois-variables-ou-plus}{%
\subsection{Tableaux croisés à trois variables ou plus}\label{tableaux-croisuxe9s-uxe0-trois-variables-ou-plus}}

\hypertarget{la-moduxe9lisation-avec-stata}{%
\chapter{La Modélisation avec Stata}\label{la-moduxe9lisation-avec-stata}}

\hypertarget{thuxe9orie-destimation}{%
\section{Théorie d'Estimation}\label{thuxe9orie-destimation}}

\hypertarget{regtession-lineaire}{%
\section{Regtession lineaire}\label{regtession-lineaire}}

:::

Read more here \url{https://bookdown.org/yihui/bookdown/markdown-extensions-by-bookdown.html}.

\hypertarget{regression-logistique}{%
\section{regression Logistique}\label{regression-logistique}}

The R Markdown Cookbook provides more help on how to use custom blocks to design your own callouts: \url{https://bookdown.org/yihui/rmarkdown-cookbook/custom-blocks.html}

\hypertarget{regresson-logistique-binoliale}{%
\subsection{Regresson logistique binoliale}\label{regresson-logistique-binoliale}}

\hypertarget{regression-logistique-multinomiale}{%
\subsection{Regression Logistique Multinomiale}\label{regression-logistique-multinomiale}}

\hypertarget{regression-logistique-ordonnuxe9}{%
\section{Regression Logistique Ordonné}\label{regression-logistique-ordonnuxe9}}

\hypertarget{analyse-des-donnuxe9es-de-logitudinales}{%
\chapter{Analyse des données de Logitudinales}\label{analyse-des-donnuxe9es-de-logitudinales}}

\hypertarget{introduction-aux-series-temporelles}{%
\section{Introduction aux series temporelles}\label{introduction-aux-series-temporelles}}

\hypertarget{donnuxe9es-de-panel}{%
\section{Données de Panel}\label{donnuxe9es-de-panel}}

\hypertarget{analuse-de-survie}{%
\section{Analuse de Survie}\label{analuse-de-survie}}

\hypertarget{analyses-exploratioires}{%
\chapter{Analyses Exploratioires}\label{analyses-exploratioires}}

\hypertarget{acp}{%
\section{ACP}\label{acp}}

\hypertarget{afc}{%
\section{AFC}\label{afc}}

\hypertarget{acm}{%
\section{ACM}\label{acm}}

  \bibliography{book.bib,packages.bib}

\end{document}
